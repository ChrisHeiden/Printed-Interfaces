\documentclass[00_doc.tex]{subfiles}
\begin{document}
    \subsection{Conclusion of the Idea and the Class}
    \begin{flushleft}
        The class Printed Interfaces helped everyone to get to know the basics in programming,
        electrtechnic, presenting ideas. This will help in future projects, and it gives anyone 
        a good set of basic knowledge, so we could learn autodidactic other topics in this field.\\~\\

        The project ended well. Sure, we couldn't achieve to use wood but this wasn't the fault of 
        the technique we used. The problem was that a diameter of 3 cm in the middle of the blossom 
        is too big if we just cut such small leaves. The leaves have to be bigger for this, so 
        they can connect and not break at the same time. Perhaps thicker wood like 2mm or 3mm would 
        have helped to achieve this. However, this could not be tested in this amount of time.
        Still, even without using wood and just using cork instead, every functionality could be 
        programmed, so the prototype can show the whole set of functionalities: change light color 
        and change brightness.
    \end{flushleft}
\end{document}
