\documentclass[00_doc.tex]{subfiles}
\begin{document}
    \section{Class Process}
    \begin{flushleft}
        To teach the class the major concepts of circuits and how to to create one with a 
        microcontroller {\footnote{\label{foot: microcontroller} Microcontrollers are extremely 
        small computers that became possible due to the progress of modern micro-technologies. 
        The major difference between computers and microcontrollers is that we can extend the 
        microcontroller by a wide range of sensors that can be used as input or output devices, 
        so it influences the real world around it. \cite{Schief1997, Dembowski2014}}}, the teacher 
        taught the basic principles of Ohm's law, voltage and amper. Furthermore, he thaught the 
        students how current flows and how to program a microcontroller. Therefore, the teacher 
        showed a circuit simulations, so students can understand the basic principles fast 
        \cite{Wegener2020}. \\~\\

        In the first session, the teacher spoke about the current flow and Ohm's law. Therefore, he 
        showed the students a virtual circuit that he could change in real-time.\\

        In the second session, he talked about microcontrollers and how these devices can be used. 
        He spent a short time explaining the different pins that can be used on an Arduino Uno 
        like the GND, 5V, 3,3, A1, D1 or PWM pins and how to program them. Therefore, he did a 
        live coding session.  He showed us how to program an LED, humidity and a capacitive sensor. 
        The humidity sensor and capacitive sensor have been drawn with conductive ink.\\

        In the third lecture, he taught the students more about programming, so it became more 
        complex. There he introduced variables and how to define and initiate them, talked about
        the different data types, functions like setup and loop and how to read and write elecricity
        with a microcontroller. In the second half of the class, he introduce InkScape. With this 
        program, we can design the conductive circuits that we can print out by a printer or 
        screen print.\\

        Later he wanted us to find project ideas that we present. So, every student presented one 
        or multiple ideas and the group talked about them. Sometimes, they discussed how to improve 
        the idea or how this idea can be realized like the material and what kind of sensors has to
        used. After the discussion with everyone, we programmed a bit more, so the students become 
        used to it. The teacher also introduced arrays and the concepts behind it. \\

        A week later, we talked and discussed the projects more detailed. This was the final class 
        before anyone started their project and build it. In the middle of the class period, we had 
        to present the project process.  \\
        At the end of the class we had to present our final prototype. Therfore, anyone prepared a 
        presenation and showed their product and its functions. Furthermore, we also had to talk about 
        the inspiration we had, and we had to talk about related work. Anyone ahd 10 minutes to talk
        about anything. Later we had an Q\&A session of 5 minutes.
    \end{flushleft}
\end{document}