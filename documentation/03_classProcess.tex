\documentclass[00_doc.tex]{subfiles}
\begin{document}
    \section{Class Process}
    \begin{flushleft}
        To teach the class the major concepts of circuits and how to create one with a 
        microcontroller {\footnote{\label{foot: microcontroller} Microcontrollers are extremely 
        small computers that became possible due to the progress of modern micro-technologies. 
        The major difference between computers and microcontrollers is that people can extend the 
        microcontroller by a wide range of sensors that can be used as input or output devices, 
        so it influences the real world around it. \cite{Schief1997, Dembowski2014}}}, the teacher 
        taught the basic principles of Ohm's law, voltage and amper. Furthermore, he thaught the 
        students about how current flows and how to program a microcontroller. Therefore, the teacher 
        showed a circuit simulations, so students can understand the basic principles fast 
        \cite{Wegener2020}. To teach all this, he used multiple sessions.\\~\\

        In the first session, the teacher spoke about the current flow and Ohm's law. Therefore, he 
        showed the students a virtual circuit that he could change in real-time.\cite{Wegener2020}\\

        In the second session, he talked about microcontrollers and how these devices can be used. 
        He spent a short time explaining the different pins that can be used on an Arduino Uno \cite{arduinoUno}
        like the GND, 5V, 3,3, A1, D1 or PWM pins and how to program them. Therefore, he did a 
        live coding session. He showed us how to program a LED, humidity and a capacitive sensor. 
        The humidity sensor and capacitive sensor have been drawn with conductive ink.\\

        In the third lecture, he taught the students more about programming, so it became more 
        complex. There he introduced variables and how to define and initiate them. Furthermore, he 
        talked about the different data types, functions like setup and loop and how to read and 
        write electricity with a microcontroller. In the second half of the class, he introduces 
        Inkscape\cite{inkscape}. With this program, designs can be made digitally. This can be used
        to create conductive circuits later (screenprint).\\

        Later he wanted us to find project ideas. So, every student presented one 
        or multiple ideas and the group talked about them. Sometimes, they discussed how to improve 
        the idea or how this idea can be realized like the material and what kind of sensors has to
        use. After the discussion with everyone, the classes programmed a bit more, so the students become 
        used to coding and microcontroller. The teacher also introduced arrays and the concepts 
        behind it. \\

        A week later, the class talked and discussed the projects more detailed. This was the final class 
        before anyone started their project and build it. In the middle of the class period, every student had 
        to present their project process.  \\

        In the end of the class, every student had to present their final prototype. Therefore, every student 
        prepared a presentation and showed their product and its functions. In this presentation, 
        students had to talk about their inspiration for this project, related work, and the user 
        group. Anyone had 10 minutes to talk about these topics. After the presentation, they had an 
        Q\&A session of 5 minutes.
    \end{flushleft}
\end{document}