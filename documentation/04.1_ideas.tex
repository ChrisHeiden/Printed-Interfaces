\documentclass[04_projectProcess.tex]{subfiles}
\begin{document}
    \subsection{Ideas}
    \label{Ideas}

    \subsubsection{Color-changing Bracelet}
    \begin{flushleft}
        Our first idea was to build and program a color-changing bracelet. The bracelet consist of several 
        parts. First, there are the main parts that include the microcontroller and the battery and other 
        sensors if needed. The second part of the device are the threads. Every thread can change its color 
        if someone of your friends has such a thread as well, and is close to you. Therefore, these bracelets 
        can create a network, so they can communicate with each other. The network strength can be used to 
        identify if a friend is close. Then it will change the color of the thread if the specific friend is
        close to the user. Also, groups can be defined. Every group has its own thread on the bracelet and 
        if someone from the group is next to you, the system changes the color of this specific thread as 
        well (see Figure ~\ref{fig:braceltIdea}).
    \end{flushleft}

    \begin{figure}[H]
        \centering
        \begin{subfigure}{.45\textwidth}
            \centering
            \includegraphics[scale=0.4]{images/projectideas/bracelt_1.png}
            \caption{Shows the bracelt concept.}
            \label{fig:braceltIdea}
            \vspace{6mm}
        \end{subfigure}
        \medskip
        \hspace{1mm}
        \begin{subfigure}{.45\textwidth}
            \centering
            \includegraphics[scale=0.4]{images/projectideas/bracelt_2.png}
            \caption{Shows the bracelt concept.}
            \label{fig:braceltIdea}
            \vspace{6mm}
        \end{subfigure}
        \caption{Shows the color-changing bracelet idea and its functionality.}
        \label{fig:drillingProcess}
    \end{figure}

    \subsubsection{Music Controll Jacket}
    \begin{flushleft}
        Anyone knows the situation, you walk around in the city and it is very cold outside. At this time, 
        you don't want to grab the smartphone often. This is necessary if the user wants to change the 
        music. In addition to this, if the users often want to change the song, he/she often has to grab
        the phone and it becomes a habit. \\
        So, another idea was to develop a jacket that can be used to change a song you are listening on a 
        phone. Therefore, strips are placed on the left or right arm that can be used to play the next or 
        previous song in a playlist, make the song louder or quieter or stop the song playing (see Figure 
        ~\ref{fig:jacketIdea}).
    \end{flushleft}

    \begin{figure}[h!]
        \centering
        \includegraphics[scale=0.2]{images/projectideas/jacket.png}
        \caption{Shows the concept of a jacket with interaction constraints.}
        \label{fig:jacketIdea}
    \end{figure}

    \subsubsection{Blossom Shaped Lamp}
    \label{BlossomShapedLamp}
    \begin{flushleft}
        A third and last idea is about lightning. The idea is to build a lamp of a blossom shape that 
        can be modified by the user. Every blossom leaf has a magnet inside, so it is possible to connect 
        each leaf. If some leaves are connected, the lamps that is in the middle of the flower blossom 
        change its color. Moreover, an analog slider will be used to change the brightness of the lamp 
        (see Figure ~\ref{fig:blossomLamp}).
    \end{flushleft}

    \begin{figure}[h!]
        \centering
        \includegraphics[scale=0.4]{images/projectideas/blossomLamp.png}
        \caption{Shows the concept of a blossom shaped lamp.}
        \label{fig:blossomLamp}
    \end{figure}
\end{document}