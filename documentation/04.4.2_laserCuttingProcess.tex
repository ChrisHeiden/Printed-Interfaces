\documentclass[04.3_buildingProcess.tex]{subfiles}
\begin{document}
    \subsubsection{Laser Cutting Process}
    \begin{flushleft}
        \noindent
        After milling the base, we had to experiment with a laser cutter to find out how big the holes and the 
        distance between them has to be so that we can laser cut the blossom shape later. To laser cut 
        everything, we used Illustator\cite{illustrator} and a laser cutter. With Illustator we can design 
        the shape that has to be cut. Therefore, we tried different widths (1mm, 0.75mm and 0.5mm) and 
        distances (1mm, 0.75mm, 0.5mm). The result can be found in Figure ~\ref{fig:01_LaserCut}.\\~\\

        \noindent
        At first, we wanted to try acrylic glass because it is transparent and it would help to light up 
        a larger space; however, after this, we decided to work with wood again because it is more flexible
        (see Figure ~\ref{fig:03_LaserCut} and ~\ref{fig:04_LaserCut}).

        \begin{figure}[H]
            \centering
            \begin{subfigure}{.45\textwidth}
            \centering
            \includegraphics[width=0.8\linewidth]{images/materialProcess/01_LaserCut.jpg}
            \caption{Shows the first try to laser cut in general, so we wanted to
                    find out what distances of the wholes and how big the wholes 
                    should be in general.}
            \label{fig:01_LaserCut}
            \vspace{6mm}
            \end{subfigure}
            \medskip
            \hspace{1mm}
            \begin{subfigure}{.45\textwidth}
                \centering
                \includegraphics[width=0.8\linewidth]{images/materialProcess/02_LaserCut.jpg}
                \caption{Shows the second try to laser cut the shape. This time we have been
                        working with acrylic glass that was 2mm thin. The wholes are 1mm thick 
                        and the distance between those are only 5mm. When we tried to move
                        the parts, they broke.}
                \label{fig:02_LaserCut}
                \vspace{6mm}
            \end{subfigure}
            \hspace{1mm}
            \begin{subfigure}{.45\textwidth}
                \centering
                \includegraphics[width=0.6\linewidth]{images/materialProcess/03_LaserCut.jpg}
                \caption{Shows the third try to laser cut the shape. This time we have been
                        working with wood plate what is 3mm thin. The wholes are 1mm thick 
                        and the distance between those are only 5mm. When we tried to move
                        the parts, they broke.}
                \label{fig:03_LaserCut}
                \vspace{6mm}
            \end{subfigure}
            \hspace{1mm}
            \begin{subfigure}{.45\textwidth}
                \centering
                \includegraphics[width=0.77\linewidth]{images/materialProcess/04_LaserCut.jpg}
                \caption{Shows the fourth try to laser cut the shape. This time we have been
                        working with MDR 1mm thin plate. The wholes are 0.5mm thick and the 
                        distance between those are only 2.5mm, so they broke.}
                \label{fig:04_LaserCut}
                \vspace{6mm}
            \end{subfigure}
            \hspace{1mm}
            \begin{subfigure}{.45\textwidth}
                \centering
                \includegraphics[width=0.6\linewidth]{images/materialProcess/06_LaserCut.jpg}
                \caption{Shows the sixth try with MDR 1mm thin plate.}
                \label{fig:04_LaserCut}
                \vspace{6mm}
            \end{subfigure}
            \begin{subfigure}{.45\textwidth}
                \centering
                \includegraphics[width=0.65\linewidth, angle=270]{images/materialProcess/08_LaserCut.jpg}
                \caption{Shows the final laser cut that shows that it is flexiable.}
                \label{fig:08_LaserCut}
                \vspace{6mm}
            \end{subfigure}
            \caption{Shows the process of laser cuttings.}
            \label{fig:laserCutTests}
        \end{figure}
    \end{flushleft}
\end{document}