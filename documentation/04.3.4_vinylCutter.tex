\documentclass[04.3_buildingProcess.tex]{subfiles}
\begin{document}
    \subsubsection{Vinyl Cutting Process}
    \noindent
    Furthermore, we used a vinyl cutter \cite{vinylCutter}  to cut our cupper foil in a 
    round shape. These will be used as capacitive sensors and for the analog slider later. The software 
    Summa Cutter Tools \cite{SoftwareSumma} has been used to draw the circles digitally. These will be used as sliders in the 
    program, so the user can change the brightness of the LEDs (see Figure \ref{fig:vinylCuttingProp}).

    \begin{figure}[H]
        \centering
        \begin{subfigure}{.45\textwidth}
            \centering
            \includegraphics[scale=0.04]{images/materialProcess/vinylCutter.jpg}
            \caption{Shows the vinyl cutter.}
            \label{fig:vinylCutter}
            \vspace{6mm}
        \end{subfigure}
        \medskip
        \hspace{1mm}
        \begin{subfigure}{.45\textwidth}
            \centering
            \includegraphics[scale=0.04]{images/materialProcess/vinylCuttingProp.jpg}
            \caption{Shows vinyl cutting test.}
            \label{fig:vinylCuttingProp}
            \vspace{6mm}
        \end{subfigure}
        \caption{Shows the vinyl cuting process.}
        \label{fig:laserCutTests}
    \end{figure}
\end{document}

