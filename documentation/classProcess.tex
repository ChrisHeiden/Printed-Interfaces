\documentclass[doc.tex]{subfiles}
\begin{document}
    \section{Class Process}
    \begin{flushleft}
        To teach the class the major concepts of circuits and to create one with a microcontroller {\footnote{\label{foot:
        microcontroller} Microcontrollers are extremly small computers that became possible due to the progress of modern 
        micro-technologies. The point is that such microcontrollers can be extended by a wide range of sensors that can 
        be used as input or output devices that can be used to influence the real world around it. \cite{Schief1997, 
        Dembowski2014}}}, the teacher taught the basic principles of Ohm's law, how the current flows and how to program 
        a microcontroller. This has been done in the first 4 sessions.\newline
        In the first session, the teacher spoke about the current flow and Ohm's law. Therefore, he showed the students
        a virtual circuit that he could change in real-time.\newline
        In the second session, he talked about microcontrollers and how these devices can be used. He spent short
        time in explaining the pins and how to program them. To do so, he showed how it is possible to program
        everything on a projector (live coding). He showed us how to program an LED, simple moisture and 
        a capacitive sensor. The simple moisture and capacitive sensor has been drawn with conductive ink.\newline
        In the third lecture, he taught the students more about programming, so it became more complex. However, 
        the main focus was to introduce InkScape. With this program, we wanted to print a conductive circuit on paper.\newline
        Later he wanted us to find project ideas that we present. So, every student presented one or multiple ideas and
        the group talked about them. Sometimes, they discussed improvements and sometimes, they talk about the idea itself
        and the way to implement and build everything. \\
        Later we talked and discussed the projects more detailed, so anyone can start programming and building everything.\\
        In the middle of the class period, we had to present the project process.  
    \end{flushleft}
\end{document}