\documentclass[04_projectProcess.tex]{subfiles}
\begin{document}
    \subsection{Soldering and Implementation Process}
    \begin{flushleft}
        After getting to know the material, we had to solder everything (see \ref{fig:solderingProcess}).
        We used an Arduino Pro Mini \cite{arduinoProMini} as a microcontroller, some resistors (like 
        1M\si{\ohm} resistors), copper foil and a Neopixel WS2812. \\      
        
        \begin{figure}[H]
            \centering
            \begin{subfigure}{.45\textwidth}
            \centering
            \includegraphics[width=0.8\linewidth]{images/programmingProcess/solderingProcess.jpg}
            \caption{Shows a testing soldering.}
            \label{fig:solderingProcess_0}
            \vspace{6mm}
            \end{subfigure}
            \begin{subfigure}{.45\textwidth}
                \centering
                \includegraphics[scale=0.03]{images/programmingProcess/Soldering_Process_1.jpg}
                \caption{Shows the soldering process of the RGB-LEDs from the top.}
                \label{fig:solderingProcess_1}
                \vspace{6mm}
            \end{subfigure}
            \hspace{1mm}
            \begin{subfigure}{.45\textwidth}
                \centering
                \includegraphics[scale=0.03]{images/programmingProcess/Soldering_Process_2.jpg}
                \caption{Shows the soldering process of the RGB-LEDs from the bottom.}
                \label{fig:solderingProcess_2}
                \vspace{6mm}
            \end{subfigure}
            \begin{subfigure}{.45\textwidth}
                \centering
                \includegraphics[width=0.8\linewidth]{images/programmingProcess/finalSoldering.jpg}
                \caption{Shows the outcome of the soldering process.} 
                \label{fig:finalSoldering}
                \vspace{6mm}
            \end{subfigure}
            \begin{subfigure}{.45\textwidth}
                \centering
                \includegraphics[width=0.8\linewidth]{images/programmingProcess/sensorLinks.png}
                \caption{Shows the links between the microcontroller and the LEDs and copper foil that is 
                used as a capacitive sensor. This image has been done by Frizing. \cite{fritzing}}
                \label{fig:solderingProcess_3}
                \vspace{6mm}
            \end{subfigure}
            \caption{Shows the soldering process and linking.}
            \label{fig:solderingProcess}
        \end{figure}

        \noindent
        To program every functionality, we used the Arduino IDE and programmed everything in C++ with 
        some additional Arduino specific function and global variables \cite{introductionArduino}. 
        We programmed different classes that specifies different functionalities or sensor \cite{
        arduinoClasses}. The following UML diagram shows the final code structure of the project (see 
        Figure \ref{fig:UMLDiagram}).

        \begin{figure}[H]
            \centering
            \includegraphics[width=0.8\linewidth]{images/programmingProcess/BlossomLamp_UML.png}
            \caption{Shows the code stucture and UML diagram of the project.}
            \label{fig:UMLDiagram}
        \end{figure}

        \noindent
        The resistors and the copper foil are used to build a capacitive sensor. \cite{Badger2019} 
        If the user pushes the capacitive sensor, the microcontroller can sense it. This functionality
        will be used to create an analog slider. We programmed the slider, so users can change the 
        brightness. However, we just programmed three capacitive sensors because the Arduino Pro Mini 
        doesn't have enough digital pins for every sensor. However, three capacitivesensors can show 
        the concept and the initial idea. The soldering process of capacitive sensors can be found in 
        Figure \ref{fig:finalSoldering}. \\~\\

        \noindent
        The Neopixel WS2812 will be used to make the lamp shine in different colors. \cite{Burgess2019} 
        To get close to the initial idea, we drilled three more whole inside the base and 
        pulled the Neopixel WS2812 inside. Sure, before we soldered cables on them, so we can 
        test them before soldering everything together. After testing the pixels, we soldered 
        them on the microcontroller (see \ref{fig:solderingProcess_1}).\\~\\

        \noindent
        Furthermore, we implemented the code for the leave connections, so the system can detect if 
        the users pull a blossom. This process changes the color of the lamp.

        \noindent
        To program these functionalities, we programmed an Arduino program that can deal with these inputs 
        and outputs. The final code can be found on GitHub. %TODO: Add right webpage
    \end{flushleft}
\end{document}
