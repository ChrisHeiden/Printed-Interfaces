\documentclass[04_projectProcess.tex]{subfiles}
\begin{document}
    \subsection{Detailed Explaination of the Blossom Shaped Lamp Idea}
        \begin{flushleft}
            In the end, we decided to develop the third idea (see chapter \ref{BlossomShapedLamp}). 
            That's why we need to explain how it should work in detail. \\~\\

            The lamp will consist of two parts. First, there will be the base that will be out of 
            wood. Inside of this base, there will be a microcontroller that regulates the whole 
            system, wires, a battery and the LEDs. Therefore, many wholes and wire paths will be 
            drilled. The base of the prototype will be milled, so we get a perfect shape. Therefore, 
            we will use Autodesk Fusion 360\cite{autodeskFusion360} to model it.\\

            The second main part will be the blossom that will be on top of the base. This blossom
            will be laser cut out of wood. A special pattern will be used, so the wood becomes flexible.
            This flexibility will be used to get a connection to each other. Every leaf defines a 
            specific color, so if one is't connected anymore, the color of the light will change. \\

            Around the blossom, there will be several capacitive sensors that can detect if one of them 
            is touched. By touching two or more capacitive sensors during a short time, tbe user can 
            define the brightness. By a clockwise interaction, the brightness becomes higher and 
            by a counterclockwise interaction, the brightness gets lower. \\

            To develop such a device, we want to use an Arduino Pro Mini, copper foil and programmable 
            WS2812B LEDs. Furthermore, we want to use a CNC machine and laser cutters to build the 
            prototype.
        \end{flushleft}
\end{document}
