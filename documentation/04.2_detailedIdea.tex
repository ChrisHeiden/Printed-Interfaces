\documentclass[04_projectProcess.tex]{subfiles}
\begin{document}
    \subsection{Detailed Explaination of the Blossom Shaped Lamp Idea}
        \begin{flushleft}
            In the end, we decided to develop the third idea (see chapter \ref{BlossomShapedLamp}). 
            That's why we want to explain how it should work in detail. \\
            The lamp will consist of two parts. First, there is the base that will be out of wood. The
            second part will be the blossom that is on top of the base. The base of the prototype will 
            be milled, so we get a perfect shape. Therefore, we will use Autodesk Fusion 360\cite{
            autodeskFusion360} to model it. On the other hand, the blossom shaped will be laser cut, 
            so the wood becomes flexible.\\
            Around the blossom, there will be several capacitive sensors that can detect if one of them 
            is touched. By touching two or more capacitive sensors in a row, we can define if the 
            brightness should be increased or lowered. This interaction will change the brightness of 
            the lamp. \\
            However, the major functionality is the blossom shaped lamp itself. There, we want to laser 
            cut the shape, so the users of the lamp can take one or several blossoms to open it. This will 
            change the color of the light. We want to use wood. Therefore, we have to cut holes inside 
            the wood, so it becomes flexible.\\
            If all blossoms are connected, then the lamp doesn't shine at all. However, if the user opens 
            the blossom, the lamp turns on and the color changes.\\
            To develop such a device, we want to use an Arduino Pro Mini, copper foil and programmable 
            WS2812B LEDs. Furthermore, we want to use a CNC machine and laser cutters to build the 
            prototype.
        \end{flushleft}
\end{document}
