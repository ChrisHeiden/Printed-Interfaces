\documentclass[00_doc.tex]{subfiles}
\begin{document}
    \section{Changes in the initial Idea}
    \begin{flushleft}
        The realization of the idea isn't the same that has been defined beforehand (see chapter 
        \ref{Ideas}). Most of the changes have to do with the idea of defining the light color 
        by every leaf. These have been the most difficult task and couldn't achieve. There was only one 
        solution to this. In the middle of the blossom, there has to be a cone or a cylinder 
        that acts like as digital reader and if a specific leaf is connected to this object, the 
        lamp has to change its color. However, this would have reduced the amount of space for 
        the light, so another idea had to be found. The solution was to use two leaves that 
        have to be connected. This idea shrinks the amount of colors that can be shown; 
        however, this would keep the design of the initial idea. \\

        This also led to a change in colors. Initially, specific leaves would have defined 
        a specific color. However, in the programming and building process, this has been 
        changed. Now the number of connections will change the color and not a specific pair
        of leaves.        
    \end{flushleft}
\end{document}
