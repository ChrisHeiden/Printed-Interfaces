\documentclass[00_doc.tex]{subfiles}
\begin{document}
    \section{Changes in the initial Idea}
    \begin{flushleft}
        The realization of the idea isn't the same that has been defined beforehand (see chapter \ref{Ideas}). 
        Most of the changes have to do with the idea to define the color by one leaf. These have been the 
        most difficult task and couldn't achieve. There was only one solution we came up with. In the middle
        of the blossom, there has to be a cone or a cylinder that acts like a digital reader and if a specific
        leaf is connected to this object, the lamp has to change its color. However, this would have 
        reduced the amount of space for the light, so we had to find another idea. The solution was to use
        two leaves that have to be connected. This leads to a shrinking amount of 
        colors that can be shown; however, this would keep the design of the initial idea. That's 
        another reason for this decision.\\

        This also leads to a change in colors. In this change, we initially thought that a specific pair 
        would define a specific color. However, in the programming and building process, we changed it 
        and programmed that the number of connections will change the color and not a specific pair
        of leaves.        
    \end{flushleft}
\end{document}
