\documentclass[04.3_buildingProcess.tex]{subfiles}
\begin{document}
    \subsubsection{Milling Process}

    \begin{flushleft}
        To build the prototype, we modeled the base for the lamp (see Figure ~\ref{fig:blossomBaseModel}). 
        Therefore, we used Autodesk Fusion 360 \cite{autodeskFusion360}. After designing the base, we used a 
        milling machine. Therefore, we only had to convert the 3D model into a ".iges"-file.
        The result of the milling process can be seen in Figure ~\ref{fig:blossomBase}.
    \end{flushleft}

    \begin{figure}[H]
        \centering
        \begin{subfigure}{.45\textwidth}
            \centering
            \includegraphics[scale=0.25]{images/materialProcess/FlowerLamp.png}
            \caption{Shows the three-dimensional digital base of the blossom lamp idea.}
            \label{fig:blossomBaseModel}
            \vspace{6mm}
        \end{subfigure}
        \hspace{1mm}
        \begin{subfigure}{.45\textwidth}
            \centering
            \includegraphics[scale=0.05]{images/materialProcess/base.jpg}
            \caption{Shows the base that has been milled.}
            \label{fig:blossomBase}
            \vspace{6mm}
        \end{subfigure}
        \caption{Show the steps we had to do to drill the base in a CNC machine.}
        \label{fig:laserCutTests}
    \end{figure}
\end{document}