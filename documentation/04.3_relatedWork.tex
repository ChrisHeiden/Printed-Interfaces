\documentclass[04_projectProcess.tex]{subfiles}
\begin{document}
    \subsection{Related Work}
        \begin{flushleft}
            Many products are related to this project because many products deal with lightning.
            Light is very important for humans.\\~\\
            It gives us a hint about the time, when we should go sleeping and when we should get up. The 
            biological rhythm helps us in our daily life. In addition to this, the color of the light helps
            us to create a form of atmosphere in certain situations.\\
            
            One product that deals with light are a common work light. It is less expensive and people can only 
            say if they want the light on or off. There are no other functionalities. Still, nearly anyone has 
            such a lamp in his/her office. \\

            However, the number of functions of these lamps is limited. That's why companies gave them 
            additional functions like changing the color. Usually, this can be done by a smartphone app. 
            There, people can define the color of the room. Furthermore, they can define a time when the lamp 
            has to switch on or off. A product that can change its color is the Philip Hue lamps. In this 
            ecosystem, people can do all this in an app.  \\ 

            Still, there is no lamp you can interact with as you do with this one. Here, it allows you to interact
            and to engage with the lamp again. Furthermore, there is no need for 
            using an app on the smartphone. So, there is less distraction, and anyone can use it. Older people
            or very young people have problems using a smartphone. People have to find the right app, have to
            understand how it works and also have to define the new colors they want to use. That's why 
            the idea of having a lamp you can interact with and to change the color and the brightness, perhaps 
            is a good one. \\~\\

            We have been inspired by a flexiable wood notebook that has been found in summer. There, the cover of 
            the the book was out of wood. If you opened it the back of the book lighted up (see Figure 
            \ref{fig:inspirationLightUp}) and by closing it the light switched off (see Figure \ref{fig:inspirationLightOff}). 
            This was the inspiration of the whole idea. We know we wanted to do something 
            with flexiable wood and with light. Because of our background, we add some interaction functionalities,
            so we came up with this idea.

            \begin{figure}[H]
                \centering
                \begin{subfigure}{.45\textwidth}
                    \centering
                    \includegraphics[width=0.8\linewidth]{images/projectideas/inspiration_on.jpg}
                    \caption{Shows the notebook open and light up.}
                    \label{fig:inspirationLightUp}
                \vspace{6mm}
                \end{subfigure}
                \begin{subfigure}{.45\textwidth}
                    \centering
                    \includegraphics[scale=0.03, angle=90]{images/projectideas/inspiration_off.jpg}
                    \caption{Shows the notebook close and light off.}
                    \label{fig:inspirationLightOff}
                    \vspace{6mm}
                \end{subfigure}
                \caption{Shows the inspiration we had.}
                \label{fig:inspiration}
            \end{figure}
        \end{flushleft}
\end{document}
