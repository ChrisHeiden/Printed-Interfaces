\documentclass[04.3_buildingProcess.tex]{subfiles}
\begin{document}
    \subsubsection{Drilling Process}
    \noindent
    Afterward, we had to drill holes inside the base, so we can link the copper foil that is 
    used for the analog slider with wires (see Figure \ref{fig:capSensorHole}). In addition to 
    this, we had to drill bigger holes inside the base. In this we want to attach the 
    microcontroller and the battery (see Figure \ref{fig:LEDHoles}). Furthermore, we drilled 
    paths inside the base, so wires can go from every whole to the microcontroller. So, users 
    should not see any wires outside of the whole installation (see Figure \ref{fig:wirePath}).

    \begin{figure}[H]
        \centering
        \begin{subfigure}{.45\textwidth}
        \centering
        \includegraphics[width=0.6\linewidth]{images/materialProcess/capSensorHole.jpg}
        \caption{Shows the wholes for the cupper foil and their wires.}
        \label{fig:capSensorHole}
        \vspace{6mm}
        \end{subfigure}
        %\hfill
        \medskip
        \hspace{1mm}
        \begin{subfigure}{.45\textwidth}
            \centering
            \includegraphics[width=0.6\linewidth]{images/materialProcess/LEDHoles.jpg}
            \caption{Shows the LED holes that has been drilled for the wires.}
            \label{fig:LEDHoles}
            \vspace{6mm}
        \end{subfigure}
        %\hfill
        \hspace{1mm}
        \begin{subfigure}{.45\textwidth}
            \centering
            \includegraphics[width=0.6\linewidth]{images/materialProcess/wirePath.jpg}
            \caption{Shows the paths that has been drilled inside of the base, so 
            users won't see any wiresing.}
            \label{fig:wirePath}
            \vspace{6mm}
        \end{subfigure}
        \caption{Shows the drilling process for the final version of the lamp.}
        \label{fig:drillingProcess}
    \end{figure}
\end{document}

