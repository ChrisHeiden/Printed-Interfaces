\documentclass[04.3_buildingProcess.tex]{subfiles}
\begin{document}
    \subsubsection{Conductive Ink Process}

    \noindent
    In addition to this, we tested the conductive ink on the wood and if there is still enough 
    conductivity. The result is that the microcontroller can notice a change; 
    however, the resistance is very high (how high? Messure!!!), so we had to use an analog pin 
    and a threshold to find out if a leave coupe is connected (see Figure \ref{fig:leaveConductiveInk}).

    \begin{figure}[h!]
        \centering
        \includegraphics[scale=0.05]{images/materialProcess/leaveTesting.jpg}
        \caption{Shows the testing of the conductive ink with wood.}
        \label{fig:leaveConductiveInk}
    \end{figure}

    \noindent
    After the conductive test (see Figure \ref{fig:leaveConductiveInk}), we wanted to paint 
    conductive ink on wood before laser cutting. Before, we asked the company if there is 
    something inside the paint that burns or is toxic. After knowing that it should be fine to 
    use the paint and later cut it. Therefore, we had to do some changes in the cutting pattern
    because the ink burns a bit stronger than the wood itself, so it burned parts of the leaves. 
    That's why we had to increase the distance between the cuts (see Figure \ref{fig:07_LaserCut}). 

    \begin{figure}[h!]
        \centering
        \includegraphics[scale=0.05]{images/materialProcess/07_LaserCut.jpg}
        \caption{Shows the result of the laser cut with capacitive ink on wood.}
        \label{fig:07_LaserCut}
    \end{figure}

    \noindent
    However, even after this success of cutting a test shape with a conductive ink, we couldn't 
    increase flexibility without increasing the size of the shape. However, we had to increase 
    the flexibility somehow because of the LEDs in the middle of the base and the light ball. 
    The diameter of the light ball is 3 centimeters. The flexibility and the size of the blossom 
    shape have to be large, so the leaves can connect and the not break. This problem couldn't 
    be solved in time, so we work with cork instead. This material is very flexiable, even 
    without cuts (see Figure \ref{fig:corkTest}).\\

    \begin{figure}[H]
        \centering
            \includegraphics[scale=0.05]{images/materialProcess/corkBlossomShape.jpg}
            \caption{Shows the corc blossom shape cut that will be used for the final prototype.}
            \label{fig:corkTest}
    \end{figure}
\end{document}