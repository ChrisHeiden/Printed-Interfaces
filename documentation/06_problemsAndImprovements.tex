\documentclass[00_doc.tex]{subfiles}
\begin{document}
    \section{Problems and Improvements}
    \begin{flushleft}
        The biggest problem was to cut the leaves. Not only the shape but also the cutting pattern 
        of the leaves have become a problem. It took a long time to find out how to cut these 
        leaves,so they bend and get flexible. However, after adding the ink layer the pattern had 
        to be changed again and the flexibility sank. This becomes a problem because the inner 
        circle of the blossom has a diameter of 3 cm. So, the leaves have to move even further than 
        its length. The result was that cork had to use instead. Cork is much more flexible, so it 
        solved the problem. \\~\\
        
        With more time and more money, the flexibility problem could have been solved. Either the 
        leaves have to become bigger, so the amount of cuts increases. Another idea is to use 
        thicker wood like 2mm or 3mm. The cuts have to be wider, perhaps. However, this could not 
        be tested in this amount of time.\\~\\

        In addition, there is also a problem with magnets. The magnets that have been used in this 
        probject are very strong. This leads to the problem that the leaves connects too easy even if 
        they are far from each other. This leads to a change in light color.\\~\\

        Another problem is the slider. Because of the round capacitive sensors, the user don't
        know how to move the finger to change the brightness. That's why, half moon shaped capacitive 
        sensors would give them a hint of how to interact with the system, so the brightness can be 
        increases or decreases.\\~\\
        
        Still, even without using wood and using cork instead, every functionality could be 
        programmed, so the prototype can show the whole set of functionalities: change light color 
        and change brightness. However, these problems should be fixed in the next generation.
    \end{flushleft}
\end{document}
