\documentclass[00_doc.tex]{subfiles}
\begin{document}
    \section{Problems and Improvements}
    \begin{flushleft}
        The biggest problem was to cut the leaves. Not only the shape, but also the cutting pattern 
        of the leaves have become a problem. It took a long time to find out how to cut these leaves,
        so they bend and get flexible. However, after adding the ink layer the pattern had to be changed
        again and the flexibility sank. This becomes a problem because the inner circle of the blossom 
        has a diameter of 3 cm. So, the leaves have to move even further than its length. 
        The result was that we had to use cork instead of wood. Cork is much more flexible, so it 
        solved the problem. Furthermore, we don't have to cut holes inside to make it flexible. \\~\\
        
        With more time and more money, we could have solved the flexibility problem. Either the leaves 
        have to become bigger, so the amount of cuts increases. Another idea is to use thicker wood
        like 2mm or 3mm. The cuts have to be wider, we guess. However, this could not be tested in 
        this amount of time.\\~\\
        
        Still, even without using wood and using cork instead, every functionality could be 
        programmed, so the prototype can show the whole set of functionalities: change light color 
        and change brightness. However, these problems should be fixed in the next generation.
    \end{flushleft}
\end{document}
